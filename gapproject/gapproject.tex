% $Id$
\documentclass[a4paper]{article}
\usepackage[swedish]{babel}
\usepackage[utf8]{inputenc}
\usepackage[hyphens]{url}
\usepackage{hyperref}
\usepackage{cleveref}
\usepackage[natbib,style=numeric-comp,maxbibnames=99,sorting=none]{biblatex}
\addbibresource{gapproject.bib}
\usepackage{authblk}

\pagestyle{plain}

\begin{document}
\title{Projekt: Analys av informationssäkerheten}
\author{Daniel Bosk}
\author{Lennart Franked}
\affil{%
  Department of Information System and Technology (IST)\\
  Mid Sweden University, Sundsvall
}
\date{\today}

\maketitle

% XXX must clarify this instruction
\section{Introduktion}

IT-säkerhet är en process.
En IT-miljö måste systematiskt granskas för att kunna upprätthålla säkerheten.
Detta är extra tydligt i de senaste årens nyheter, då ett flertal stora globala 
företag blev uppmärksammade den hårda vägen på att de hade stora brister 
i säkerheten~\cite[se][]{Fisher2011rsa,Hunt2011abs,Cluley2012twp,%
  Oberheide2010bao,Cubrilovic2009rhf,%
  Jones2013targethack,Ducklin2014targethack,Zetter2014targethack%
}.

Syftet med uppgiften är att du skall fördjupa dina kunskaper i analys av 
informationssäkerheten i en verksamhet.
Till din hjälp har du kurslitteraturen för att utforma din analys.
Du ska genomföra en gapanalys av informationssäkerheten på ett företag 
alternativt en avdelning på ett större företag eller organisation.


\section{Mål}

Målet med projektet är:
\begin{itemize}
	% $Id$
\item develop an understanding of and gain practical experiences to design a
classification model.
\item  implement an organisational analysis as well as a risk analysis.

\end{itemize}


\section{Genomförande}
\noindent
Du ska genomföra en del av en gapanalys av hela eller en del av företagets 
verksamhet.
Du ska alltså identifiera risker som finns i företagets informationshantering.
Därefter ska du ta fram ett förslag på åtgärder som företaget kan använda sig 
av i sitt vidare arbete med informationssäkerhet.

Utgå från den del av ISO 27000 som finns som bilaga i MSB:s
ramverk~\cite{MSB2011gb}. 
Du väljer ett av kapitlen och genomför gapanalysen bara för det kapitlet. (Om
du väljer ett av de väldigt korta kapitlen, då måste du naturligtvis välja något
fler.)


\section{Examination}

Sammanställ en rapport för din gapanalys.
Ett försättsblad innehållandes följande formalia måste finnas med:
kurskod och -namn, rapporttitel, författarnas namn samt datum.

Rapporten skall därefter följa exemplet på mall i MSB:s metodstöd \citep[bilaga 
C]{MSB2011gap}.
Rapporten ska vara utformad för att lämnas till företaget för användning i sitt
arbete med informationssäkerheten.

Rapporten ska skrivas i två utföranden:
\begin{itemize}
  \item Rapporten i dess orginalutförande lämnas till företaget.
  \item En anonymiserad version av denna rapport (kopplingar till företaget
    skall plockas bort i rapporten) som lämnas in i lärplattformen.
\end{itemize}


% XXX muntlig presentation av projekt?
%Projektet redovisas även muntligen i slutet av kursen.
%Redovisningen ska vara av akademisk karaktär, tydlig och med slides, och vara 
%cirka 10 minuter.
%Det är obligatoriskt att styrka sin identitet med legitimation vid 
%presentationstillfället.

Uppgiften kan lösas enskilt alternativt i grupp om två personer.
Om ni väljer att lösa uppgiften i grupp måste ni själva försöka ordna någon
att samarbeta med.
%Examinator kommer att välja en i gruppen som får genomföra hela presentationen, 
%kom väl förberedda.


\printbibliography{}
\end{document}
