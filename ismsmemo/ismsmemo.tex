% $Id$
\documentclass[a4paper]{llncs}
\usepackage[utf8]{inputenc}
\usepackage[swedish]{babel}
\usepackage{csquotes}
\usepackage[hyphens]{url}
\usepackage{hyperref}
\usepackage{cleveref}
\usepackage[natbib,style=numeric-comp,maxbibnames=99,sorting=none]{biblatex}
\addbibresource{ismsmemo.bib}

\usepackage{verbatim}
\let\solution\comment%

\pagestyle{plain}

% XXX translate to english, add enisa literature

\begin{document}
\title{Memo: Ledningssystem för informationssäkerhet}
\author{%
  Carina Bengtsson
}
\institute{%
  Department of Information and Communication Systems\\
  Mid Sweden University, Sundsvall
}
\date{\today}

\maketitle


\section{Introduktion}
\label{sec:introduction}
\noindent
Denna uppgift fokuserar på ledningssystem för informationssäkerhet (LIS), det 
vill säga det system som leder en verksamhets informationssäkerhetsarbete.


\section{Syfte}
\label{sec:aim}
\noindent
Syftet med uppgiften är:
\begin{itemize}
  % $Id$
\item develop an understanding of and gain practical experiences to design a
classification model.
\item  implement an organisational analysis as well as a risk analysis.

\end{itemize}


\section{Läsanvisningar}
\noindent
% $Id$
% Author:	Daniel Bosk <daniel.bosk@miun.se>
Du ska inför skrivningen av detta PM ha läst dokumenten
\begin{itemize}
  \item \emph{Introduktion till metodstödet}~\cite{MSB2011itm},
  \item \emph{Säkra ledningens engagemang}~\cite{MSB2011sle}, och
  \item \emph{Projektplanering}~\cite{MSB2011p}.
\end{itemize}



\section{Genomförande}\label{Work}
\noindent
Genomförandet av uppgiften består av fyra deluppgifter.
En inledande del om ledningssystem, gällande författningar i Sverige, 
framgångsfaktorer och en engagerad ledning.

\subsection{Ledningssystem och ledningssystem för informationssäkerhet}
\noindent
I en kort text om maximalt 350 ord, beskriv vad ett ledningssystem och ett 
ledningssystem för informationssäkerhet (LIS) är.
Beskriv också hur ett LIS fungerar och är uppbyggt.

\subsection{Krav enligt MSBFS 2009:10}
\noindent
I ett ledningssystem för informationssäkerhet ingår vissa arbetsuppgifter för 
verkamheten.
Myndigheten för samhällsskydd och beredskap (MSB) har i sin författningssamling 
MSBFS 2009:10~\cite{MSBFS2009:10}, som är bindande för statliga myndigheter, 
fastställt i 4 § att myndigheterna ska:
\begin{enumerate}
  \item upprätta en informationssäkerhetspolicy och andra styrande dokument,
  \item utse en eller flera personer som leder och samordnar arbetet med 
    informationssäkerhet,
  \item klassificera sin information med utgångspunkt i krav på 
    konfidentialitet, riktighet och tillgänglighet,
  \item använda riskanalyser för att bestämma hur risker ska hanteras samt 
    vilka åtgärder som ska vidtas, och
  \item dokumentera viktiga granskningar och säkerhetsåtgärder.
\end{enumerate}
Vidare föreskrivs i 5 § att
\begin{enumerate}\setcounter{enumi}{5}
  \item ledningen ska hålla sig informerad om arbetet och minst en gång per år 
    följa upp och utvärdera informationssäkerhetsarbetet.
\end{enumerate}

Reflektera kortfattat punkterna 1, 2 och 3 \emph{samt} en av punkterna 4, 
5 eller 6.
Skriv ner varför du tycker att respektive krav är viktiga alternativt mindre 
viktiga att genomföra i en organisation.
Observera att din tänkta organisation ej behöver vara statlig myndighet.
Du skriver maximalt 150 ord för respektive punkt.

\subsection{Framgångsfaktorer}
\noindent
Att implementera ett ledningssystem i en verksamhet är ett omfattande arbete.
I metodstödet framkommer vissa framgångsfaktorer för att lyckas.
Förklara dessa kortfattat om maximalt 150 ord totalt.

\subsection{En engagerad ledning}
\noindent
Vilka argument skulle du använda för att motivera en ledning till att fatta 
beslut om att implementera ett LIS\@?
Skriv ner minst tre argument och motivera varför.
Dessa bör omfatta maximalt 150 ord totalt.


\section{Examination}
\label{sec:examination}
\noindent
Denna uppgift kommer att examineras genom ett skriftligt PM som lämnas in som 
ett PDF-dokument.
Detta ska innehålla uppgifterna från \cref{Work}:
\begin{exercise}
  Ledningssystem och Ledningssystem för informationssäkerhet (350 ord);
  \begin{solution}
    Betyg: F-C.

    Ledningssystemet beskrivs i \emph{Introduktion till 
      metodstödet}~\cite{MSB2011itm}, med början i avsnitt 1.2 
    \enquote{Ledningssystem för informationssäkerhet}.
    Bilaga A i samma dokument förklarar också LIS på ett bra sätt.
    Ett ledningssystem är kort:
    \begin{itemize}
      \item Ett \enquote{system} för att leda verksamheten.
      \item Andra exempel: kvalitetsledningssystem, miljöledningssystem.
    \end{itemize}

    Ett ledningssystem för informationssäkerhet är ett specialfall av 
    ledningssystem, i korthet:
    \begin{itemize}
      \item Ett LIS styr verksamhetens informationssäkerhet.
        Förklarar också \emph{hur} man arbetar för att hålla ordning på 
        informationssäkerheten.
      \item Integreras med övrigt ledningssystem.
      \item Finns internationella standarder för detta: ISO 27000-serien.
      \item Metodstödet beskriver hur man bygger upp ett LIS som en PDCA-cykel 
        (förbättringscykel) -- strävar mot ständiga förbättringar.
      \item Ett LIS ska anpassas till verksamhetens behov -- metodstödet ska 
        ses som ett smörgåsbord.
      \item Förordningen MSBFS 2009:10~\cite{MSBFS2009:10} beskriver 
        skall-kraven för myndigheters LIS\@.
    \end{itemize}
  \end{solution}
  \end{exercise}

  \begin{exercise}
  Krav enligt MSBFS 2009:10, punkterna 1--3 samt en av punkterna 4--6 (600 
  ord);
  \begin{solution}
    Betyg: F eller P.

    Se MSBFS 2009:10~\cite{MSBFS2009:10}.
  \end{solution}
\end{exercise}

  \begin{exercise}
  Framgångsfaktorer (150 ord); samt
  \begin{solution}
    Betyg: F-C.

    Beskrivs i \emph{Introduktion till metodstödet}~\cite{MSB2011itm}, avsnitt 
    2 \emph{Framgångsfaktorer}.
    De framgångsfaktorer som ska beskrivas är
    \begin{itemize}
      \item ledningens engagemang,
      \item förankring i och kunskap om organisationen,
      \item tillräckliga resurser, och
      \item verksamhetsanpassning.
    \end{itemize}

    De ska förklara något från alla delar för att få full pott.
    (Det är nog lättast om du själv läser och får en uppfattning om hur de ska 
    beskriva framgångsfaktorerna -- ganska lite text så det är svårt att 
    sammanfatta i punktform.)
    Dra av ett betygssteg för varje faktor som är otillräcklig.
  \end{solution}
\end{exercise}

\begin{exercise}
  En engagerad ledning (150 ord).
  \begin{solution}
    Betyg: F eller P.

    Betyg ska inte baseras på om de anger MSB:s förslag eller ej.
    Värdera hur de resonerar kring antingen MSB:s eller deras egna förslag.
    Om de väljer att använda sig av MSB:s förslag hittas de i \emph{Säkra 
    ledningens engagemang}~\cite{MSB2011sle}, avsnitt 3.2 \emph{Argument för 
    att införa ett LIS i organisationen}.
  \end{solution}
\end{exercise}

Totalt ska ditt PM omfatta mellan 1000--1250 ord, vara skrivet med akademisk 
svenska eller engelska och ha korrekta referenser.


\printbibliography{}
\end{document}
