\documentclass[a4paper]{llncs}
\usepackage[utf8]{inputenc}
\usepackage[swedish]{babel}
\usepackage{csquotes}
\usepackage[hyphens]{url}
\usepackage{hyperref}
\usepackage[swedish]{cleveref}
\usepackage[single]{acro}
\usepackage[natbib,style=numeric-comp,maxbibnames=99,sorting=none]{biblatex}
\addbibresource{ismsmemo.bib}

\usepackage{verbatim}
\let\solution\comment%

\pagestyle{plain}

\DeclareAcronym{isms}
{
  short = ISMS,
  short-indefinite = an,
  long = information security management system,
  long-indefinite = an,
}

% XXX translate to english, add enisa literature

\begin{document}
\title{PM\@: Information Security Management System}
\author{%
  Carina Bengtsson and Lennart Franked
}
\institute{%
  Department of Information Systems and Technologies\\
  Mid Sweden University, Sundsvall
}
\date{\today}

\maketitle


\section{Introduction}
\label{sec:introduction}

This assignments focus on \acp{isms}. The purpose of \iac{isms} is to manage
an organizations work within the field of information security.

\section{Aim}
\label{sec:aim}

This assignment aims to:
\begin{itemize}
  % $Id$
\item develop an understanding of and gain practical experiences to design a
classification model.
\item  implement an organisational analysis as well as a risk analysis.

\end{itemize}


\section{Reading assignment}

% $Id$
% Author:	Daniel Bosk <daniel.bosk@miun.se>
Du ska inför skrivningen av detta PM ha läst dokumenten
\begin{itemize}
  \item \emph{Introduktion till metodstödet}~\cite{MSB2011itm},
  \item \emph{Säkra ledningens engagemang}~\cite{MSB2011sle}, och
  \item \emph{Projektplanering}~\cite{MSB2011p}.
\end{itemize}



\section{Task}\label{Work}

This assignment contains four sub tasks, One introduction task covering
\acp{isms}, including Swedish statutes, success stories and how to get a
committed leadership.

\subsection{Information Security Management System}
Shortly describe a management system is, and more specifically what \iac{isms} 
is. Your description should also cover how \iac{isms} is
structured.

\subsection{Legal requirements according to MSBFS 2009:10}

In \iac{isms}, there are certain work tasks defined for the organisation.
The Swedish Civil Contingencies Agency (MSB) have defined in their statutes
MSBFS 2009:10~\cite{MSBFS2009:10}, which is binding for governmental agencies, 
stated that the governmental agencies should:
\begin{enumerate}
  \item To build an information security policy and other policy documents,
  \item elect one or several persons to lead and coordinate the work with
    information security,
  \item classify their information with a starting point in regard to
    confidentiality, integrity and availability,
  \item with the help of risk analysis, provide a risk evaluation, estimation
    and treatment, and
  \item document important audits and security measurements.
\end{enumerate}
Further more, in 5§ that

\begin{enumerate}\setcounter{enumi}{5}
  \item the management should keep itself up to date with the work with
    information security, and at least once a year do a follow-up and evaluate
    the information security work.
\end{enumerate}

Shortly reflect around the key points 1, 2 and 3, \emph{then} select one of the key
points 4, 5 or 6.

If you are using the English course material, refer to the key-points a-i in
\cite[chap. 3.2.1]{iso27000}

Write down why you think each of the requirement is important, or not as
important to incorporate in an organisation.

Note that the organisation does not necessarily need to be a governmental
agency.

\subsection{Success factors}

To implement \iac{isms} in an organisation is a large project. Therefore
the material mentions some success factors that are critical to achieve a
successful implementation \cite[3.6]{iso27000}. Give a short summary about
these success factors.

\subsection{Commitment from the Management}
If you where put in the position to try to convince the management in a company
to implement a \iac{isms}. How would you motivate them?

\section{Examination}
\label{sec:examination}

This assignment will be examinated through a written PM that will be handed in
as a PDF-document.
The PM \emph{must} must cover the tasks from \cref{Work}:

\begin{enumerate}
  \item Information Security Management System;
    \begin{solution}
      Grade: F-C.
      \acp{isms} is described in \emph{Introduktion till 
        metodstödet}~\cite{MSB2011itm} or \cite[chap. 3]{iso27000},

      A management system can briefly be defined as:
      \begin{itemize}
        \item A \enquote{system} to manage the organization.
        \item Other example: Quality Management system, Environmental Management.
      \end{itemize}

      \Iac{isms} is a special case of a management system that cen briefly be
      explained as:
      \begin{itemize}
        \item \Iac{isms} is controlling the organizations information
          security. It also explains \emph{how} to work to manage the
          information security.

        \item Integrated with other management systems.
        \item There is an international standard for this: ISO 27000-series.
        \item MSBs 'Metodstöd' gives a description how to build \iac{isms} as
          an PDCA life cycle.
        \item \Iac{isms} should be adapted to the organizations need. MSB's
          \enquote{metodstöd} should be seen as a smorgasbord.
        \item The regulation MSBFS 2009:10~\cite{MSBFS2009:10} list what
          governmental agencies must implement, in regards to \acp{isms}.
      \end{itemize}
    \end{solution}

  \item Requirements according to MSBFS 2009:10, points 1--3 and one of the
    following key points 4--6. Or if you are using the english material,
    keypoints a-c and one the points d-i in \cite[chap. 3.2.1]{iso27000}.
    \begin{solution}
      Grade: F or P.

      Se MSBFS 2009:10~\cite{MSBFS2009:10}.
    \end{solution}

  \item Success factors; and
    \begin{solution}
      Grade: F-C.

      Described in \emph{Introduktion till metodstödet}~\cite{MSB2011itm}, avsnitt 
      2 \emph{Framgångsfaktorer}, or \cite[3.6]{iso27000}
      Success factors that needs to be mentioned are
      De framgångsfaktorer som ska beskrivas är
      \begin{itemize}
        \item Commitment from the management,
        \item Support through the entire organization,
        \item sufficient resources, and
        \item adapting the \ac{isms} to the organization.
      \end{itemize}
    \end{solution}

  \item Commitment from the Management
    \begin{solution}
      Grade: F eller P.
      
      Grading is not done solely on the student naming the suggestion given in
      the course literature, instead, it should be based on their own
      reflections.
    \end{solution}
\end{enumerate}

Your PM should be written with an academic language in either english or
swedish. It must contain correct references.

\printbibliography{}
\end{document}
