\documentclass[a4paper]{llncs}
\usepackage[utf8]{inputenc}
\usepackage[swedish]{babel}
\usepackage[hyphens]{url}
\usepackage{hyperref}
\usepackage[swedish, english]{cleveref}
\usepackage{marginnote}
\usepackage{framed}
\usepackage{booktabs}
\usepackage{xcolor}
\usepackage{pdfpages}

\usepackage[natbib,style=numeric-comp,sorting=none,maxbibnames=99]{biblatex}
\addbibresource{risksem.bib}

\usepackage{verbatim}
\let\solution\comment%

\pagestyle{plain}

% XXX translate to english, add enisa literature

\begin{document}
\title{Seminar: Organisational and risk analysis}
\author{%
  Carina Bengtsson
  \and
  Lennart Franked
}
\institute{%
  Department of Information and Communication Systems\\
  Mid Sweden University, Sundsvall
}
\date{\today}

\maketitle

\section{Introduction}
\label{sec:introduction}

The focus of this seminar is to discuss the importance of having a
classification model that is adapted to the organisation, and to practically
perform a simple organisation and risk analysis.
The seminar will be based upon the classification model that you are developing
for the seminar.

\section{Aim}
\label{sec:aim}

The aim of this assignment is to:
\begin{itemize}
  % $Id$
\item develop an understanding of and gain practical experiences to design a
classification model.
\item  implement an organisational analysis as well as a risk analysis.

\end{itemize}


\section{Reading instructions}

% $Id$
% Author:	Daniel Bosk <daniel.bosk@miun.se>
Du ska inför skrivningen av detta PM ha läst dokumenten
\begin{itemize}
  \item \emph{Introduktion till metodstödet}~\cite{MSB2011itm},
  \item \emph{Säkra ledningens engagemang}~\cite{MSB2011sle}, och
  \item \emph{Projektplanering}~\cite{MSB2011p}.
\end{itemize}



\section{Developing the classification model}
\label{sec:work}

This assignment have been divided into two parts.
The first part covers developing your own classification model. Then in the
second part you will use this model to classify some informational assets.

\subsection{Develop a classification model}
\label{sec:develop}

Develop a simple classification model that is adapted after the organisation you
have been given.
\begin{itemize}
  \item University (government),
  \item Municipality, or
  \item Travel agency (privat business).
\end{itemize}
\begin{framed}\noindent
  Hint:
  \begin{itemize}
    \item Start by identifying a couple of internal and external requirements.
      Internal requirements can be visions, policies etc. External requirements
      can be some legal requirements that the organisation needs to follow.
      To identify these requirements, you can use internet, contact
      organisations, discuss with your classmates, or make up your own.
      It is however important that you can motivate your requirements you come
      up with. You also have the lecture slides and the methodological support
      to your help.

  \end{itemize}
\end{framed}

\subsection{Using the classification model}
\label{sec:use}
\noindent
With the help of your newly developed classification model, classify the
following five informational assets:
\begin{enumerate}
  \item Salary lists,
  \item decision basis regarding the organisation that has been sent using SMS.
  \item working material for next years budget.
  \item Logs from the access control system.
  \item \enquote{customer records}; course participants for the university,
    class list on a public school, traveler records for the latest Mexico trip
    for the travel agency.
\end{enumerate}
\begin{framed}\noindent
  Hint: When you present this part of the assignment, it is recommended that you
  write it according to:
  \begin{center}
    \begin{tabular}{ll}
      \toprule
      \textbf{Area} & \textbf{Classification} \\
      \midrule
      Salary list  & C\(x\) \\
                  & A\(y\) \\
                  & I\(z\) \\
      \midrule
      Decision basis & C\(x^\prime\) \\
                      & A\(y^\prime\) \\
                      & I\(z^\prime\) \\
      \midrule
      \dots \\
      \bottomrule
    \end{tabular}
  \end{center}
  Where \(x\), \(y\) and \(z\) are the chosen levels for confidentiality,
  availability and integrity.
\end{framed}

You should motivate how you reasoned when classifying each informational assets.

\subsection{Reflection}
\label{sec:reflection}
\noindent
Reflect about your classification model. What problems did you have? Which parts
did you think was fairly straightforward. What are some problems with your
model? What are the strenght? Compare your model to the generic MSB
classification model.

\subsection{Examination}
Sections \ref{sec:develop}, \ref{sec:use} and \ref{sec:reflection} should be
written down as a memo and submitted into the course plattform in PDF-format.

Your memo must be written in academic swedish or english, and contain proper
references.

Since your might later in \ref{sec:present} have to present your model to the
class, you should also create a presentation that includes your classification
model. This must be submitted at latest the day before the seminar S3. Your
presentation should be around 5 minutes.

\section{Preparation for the seminar}

The seminar is planned to take around 4 hours and is divided into three parts;
discussions about your classification models, using classification models and
performing risk analysis on some information assets that we will identify during
the seminar.

\subsection{Discussing the classification models.}
\label{sec:present}

For each of the three organisations, we will discuss 1 - 3 different
classification models, that is, up to 9 students might present their
classification model.

\subsection{Using the classification model}
\label{sec:use}

Before we can use the classification model, you will in groups identify some
information assets based on a process mapping that we will do togheter.

You will then in the same groups classify these informational assets using a
common classification model (it will be one models that were presented in
\ref{sec:use}).

Selected groups will then present their findings.

\subsection{Risk analysis}
\label{sec:risk}

In the same groups that you were working in during \ref{sec:use}, you will
perform a risk analysis on five informational assets. You will in your groups
decide which five assets that should be subjected to the risk analysis.

For each informational asset, your group must reach to at least five threats,
along with a suggestion on how to protect this asset towards this threat.

Each group will then select two of these threats and present these for the
rest of the class.

\subsection{Examination}
\label{sec:examination}

In order to pass this part of the assignment, you need to actively participate
in the seminar. This includes that you have at latest the day before the seminar
sent in your presentation slides.

\printbibliography{}

\end{document}
