Records and archives management deals with certain kinds of information that is 
related to business processes, and serve as evidence of activities.
Why it can for example be used for accountability purposes, contracts, regulate 
business relations and more.
Therefore it is important to ensure the quality of the information, and that it 
is not manipulated for example.
The trustworthiness of the information is central, and development of criteria 
and practices to ensure that.
The emphasis is on the information, and also to understand the context in which 
the information is created and managed.
Business process analysis is therefore a central activity.
The National Archives of Sweden and the Swedish Civil Contingencies Agency has 
for example had some collaboration in that area.

The lecture will be an introduction to archives and information science, basic 
concepts, processes, business process analysis and information mapping.
More concretely, after this session you should be able to
\begin{itemize}
  \item \dots
\end{itemize}

The material covered is primarily from \citetitle{infokartl}~\cite{infokartl} 
and the standard ISO 30300:2011~\cite{ISO30300:2011}.
